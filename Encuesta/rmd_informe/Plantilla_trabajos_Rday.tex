\documentclass[]{article}
\usepackage[left=1in,top=1in,right=1in,bottom=1in]{geometry}
\newcommand*{\authorfont}{\fontfamily{phv}\selectfont}
\usepackage{lmodern}


\usepackage[T1]{fontenc}
\usepackage[utf8]{inputenc}

%------------------- Adicionales ---------------------------------

\usepackage{fancyhdr} % activamos el paquete
\pagestyle{fancy} % seleccionamos un estilo

\renewcommand{\headrulewidth}{0pt}
\fancyhead[L]{}
\fancyhead[R]{
\includegraphics[height=1.2cm]{logo_Rday}
}

%\usepackage[spanish,es-tabla]{babel} % Para activar espanol
%\selectlanguage{spanish}

\pagenumbering{gobble} % Para eliminar todos los numeros de pagina

% ----------------------------------------------------------------


\usepackage{abstract}
\renewcommand{\abstractname}{}    % clear the title
\renewcommand{\absnamepos}{empty} % originally center

\renewenvironment{abstract}
 {{%
    \setlength{\leftmargin}{0mm}
    \setlength{\rightmargin}{\leftmargin}%
  }%
  \relax}
 {\endlist}

\makeatletter
\def\@maketitle{%
  \newpage
%  \null
%  \vskip 2em%
%  \begin{center}%
  \let \footnote \thanks
    {\fontsize{18}{20}\selectfont\raggedright  \setlength{\parindent}{0pt} \@title \par}%
}
%\fi
\makeatother




\setcounter{secnumdepth}{0}

\usepackage{color}
\usepackage{fancyvrb}
\newcommand{\VerbBar}{|}
\newcommand{\VERB}{\Verb[commandchars=\\\{\}]}
\DefineVerbatimEnvironment{Highlighting}{Verbatim}{commandchars=\\\{\}}
% Add ',fontsize=\small' for more characters per line
\usepackage{framed}
\definecolor{shadecolor}{RGB}{248,248,248}
\newenvironment{Shaded}{\begin{snugshade}}{\end{snugshade}}
\newcommand{\AlertTok}[1]{\textcolor[rgb]{0.94,0.16,0.16}{#1}}
\newcommand{\AnnotationTok}[1]{\textcolor[rgb]{0.56,0.35,0.01}{\textbf{\textit{#1}}}}
\newcommand{\AttributeTok}[1]{\textcolor[rgb]{0.77,0.63,0.00}{#1}}
\newcommand{\BaseNTok}[1]{\textcolor[rgb]{0.00,0.00,0.81}{#1}}
\newcommand{\BuiltInTok}[1]{#1}
\newcommand{\CharTok}[1]{\textcolor[rgb]{0.31,0.60,0.02}{#1}}
\newcommand{\CommentTok}[1]{\textcolor[rgb]{0.56,0.35,0.01}{\textit{#1}}}
\newcommand{\CommentVarTok}[1]{\textcolor[rgb]{0.56,0.35,0.01}{\textbf{\textit{#1}}}}
\newcommand{\ConstantTok}[1]{\textcolor[rgb]{0.00,0.00,0.00}{#1}}
\newcommand{\ControlFlowTok}[1]{\textcolor[rgb]{0.13,0.29,0.53}{\textbf{#1}}}
\newcommand{\DataTypeTok}[1]{\textcolor[rgb]{0.13,0.29,0.53}{#1}}
\newcommand{\DecValTok}[1]{\textcolor[rgb]{0.00,0.00,0.81}{#1}}
\newcommand{\DocumentationTok}[1]{\textcolor[rgb]{0.56,0.35,0.01}{\textbf{\textit{#1}}}}
\newcommand{\ErrorTok}[1]{\textcolor[rgb]{0.64,0.00,0.00}{\textbf{#1}}}
\newcommand{\ExtensionTok}[1]{#1}
\newcommand{\FloatTok}[1]{\textcolor[rgb]{0.00,0.00,0.81}{#1}}
\newcommand{\FunctionTok}[1]{\textcolor[rgb]{0.00,0.00,0.00}{#1}}
\newcommand{\ImportTok}[1]{#1}
\newcommand{\InformationTok}[1]{\textcolor[rgb]{0.56,0.35,0.01}{\textbf{\textit{#1}}}}
\newcommand{\KeywordTok}[1]{\textcolor[rgb]{0.13,0.29,0.53}{\textbf{#1}}}
\newcommand{\NormalTok}[1]{#1}
\newcommand{\OperatorTok}[1]{\textcolor[rgb]{0.81,0.36,0.00}{\textbf{#1}}}
\newcommand{\OtherTok}[1]{\textcolor[rgb]{0.56,0.35,0.01}{#1}}
\newcommand{\PreprocessorTok}[1]{\textcolor[rgb]{0.56,0.35,0.01}{\textit{#1}}}
\newcommand{\RegionMarkerTok}[1]{#1}
\newcommand{\SpecialCharTok}[1]{\textcolor[rgb]{0.00,0.00,0.00}{#1}}
\newcommand{\SpecialStringTok}[1]{\textcolor[rgb]{0.31,0.60,0.02}{#1}}
\newcommand{\StringTok}[1]{\textcolor[rgb]{0.31,0.60,0.02}{#1}}
\newcommand{\VariableTok}[1]{\textcolor[rgb]{0.00,0.00,0.00}{#1}}
\newcommand{\VerbatimStringTok}[1]{\textcolor[rgb]{0.31,0.60,0.02}{#1}}
\newcommand{\WarningTok}[1]{\textcolor[rgb]{0.56,0.35,0.01}{\textbf{\textit{#1}}}}

\usepackage{graphicx,grffile}
\makeatletter
\def\maxwidth{\ifdim\Gin@nat@width>\linewidth\linewidth\else\Gin@nat@width\fi}
\def\maxheight{\ifdim\Gin@nat@height>\textheight\textheight\else\Gin@nat@height\fi}
\makeatother
% Scale images if necessary, so that they will not overflow the page
% margins by default, and it is still possible to overwrite the defaults
% using explicit options in \includegraphics[width, height, ...]{}
\setkeys{Gin}{width=\maxwidth,height=\maxheight,keepaspectratio}




\author{}


\date{}

\usepackage{titlesec}

\titleformat*{\section}{\normalsize\bfseries}
\titleformat*{\subsection}{\normalsize\itshape}
\titleformat*{\subsubsection}{\normalsize\itshape}
\titleformat*{\paragraph}{\normalsize\itshape}
\titleformat*{\subparagraph}{\normalsize\itshape}


\usepackage{natbib}
\bibliographystyle{plainnat}
\usepackage[strings]{underscore} % protect underscores in most circumstances



\newtheorem{hypothesis}{Hypothesis}
\usepackage{setspace}

\makeatletter
\@ifpackageloaded{hyperref}{}{%
\ifxetex
  \PassOptionsToPackage{hyphens}{url}\usepackage[setpagesize=false, % page size defined by xetex
              unicode=false, % unicode breaks when used with xetex
              xetex]{hyperref}
\else
  \PassOptionsToPackage{hyphens}{url}\usepackage[unicode=true]{hyperref}
\fi
}

\@ifpackageloaded{color}{
    \PassOptionsToPackage{usenames,dvipsnames}{color}
}{%
    \usepackage[usenames,dvipsnames]{color}
}
\makeatother
\hypersetup{breaklinks=true,
            bookmarks=true,
            pdfauthor={},
             pdfkeywords = {},  
            pdftitle={},
            colorlinks=true,
            citecolor=blue,
            urlcolor=blue,
            linkcolor=magenta,
            pdfborder={0 0 0}}
\urlstyle{same}  % don't use monospace font for urls

% set default figure placement to htbp
\makeatletter
\def\fps@figure{htbp}
\makeatother



% add tightlist ----------
\providecommand{\tightlist}{%
\setlength{\itemsep}{0pt}\setlength{\parskip}{0pt}}

\begin{document}
	
% \pagenumbering{arabic}% resets `page` counter to 1 
%




\vskip 6.5pt


\noindent  \hypertarget{introduccion}{%
\section{Introducción}\label{introduccion}}

Debemos empezar por dejar de rotular a los que cambian de opinión como
desertores. Más bien, son decepcionados y habría que pensar en acciones
que mejoran la atención del estudiante. \citet{devries2011dod} La
deserción estudiantil se define según el ministerio de educación
nacional como `` Estado de un estudiante que de manera voluntaria o
forzosa no registra matrícula por dos o más períodos académicos
consecutivos del programa en el que se matriculó; y no se encuentra como
graduado, o retirado por motivos disciplinarios''
\citep{MinisteriodeEducacionNacional}. Para hablar de deserción
estudiantil y sus causas es pertinente conocer cómo se organiza el
sistema de educación superior en Colombia, el cual esta divido en dos
niveles conocidos como pregrado y posgrado, el pregrado a su vez se
divide en tres niveles de formación que son: Técnico profesional,
Tecnológico y Profesional, y el posgrado se divide en Maestrías y
doctorados \citep{MinisteriodeeducacionNacionala}. La deserción
estudiantil en educación superior en Colombia es documentada y
consolidada por el ministerio de educación a través del sistema para la
Prevención de la Deserción en las Instituciones de Educación Superior
``SPADIES'' \citep{MinisteriodeeducacionNacional1}, el cual permite
conocer y registrar aspectos socioeconómicos, académicos,
institucionales e individuales de los estudiantes con el fin de estudiar
este fenómeno. La deserción a través del SPADIES es analizada de dos
maneras, por periodo académico ``desertores un año después matriculados
dos semestres atrás'' y por cohorte; las tasas más altas registradas
para los años 2015 y 2016 es para la técnica profesional ,mientras que
la más baja es para el nivel universitario; Sin embargo, las cifras son
alarmantes en todos los niveles de educación superior ya que la tasa de
deserción por cohorte de todos los niveles de educación ronda entre un
45 \% y un 57 \% \citep{SPADIES2017}. Para el nivel universitario en
Colombia se registra una tasa de deserción estudiantil por cohorte de
46,1\% en el año 2015 y en Antioquia para el mismo año se registra una
tasa de 47,1\% \citep{SPADIES2016}. La tasa de deserción por perdido
para el año 2015 en Colombia para el nivel universitario fue de 9,25 \%
y a nivel de Antioquia se registró una deserción por periodo en el nivel
universitario de 9,55\% \citep{MinisteriodeEducacionNacional2016}.
Antioquia es el segundo departamento con más instituciones de educación
superior, contando a la fecha de 2015 con 52 instituciones, 15 de ellas
de carácter oficial, 35 de carácter privado y 2 de régimen especial,
oferta un total de 1.907 programas curriculares de los cuales 629
corresponden a programas universitarios, captura aproximadamente el 15\%
de la población de educación superior inscrita del país y gradúa
aproximadamente el 13\% de estudiantes de educación superior de Colombia
\citep{MinisteriodeEducacionNacional2016}. Medellín siendo la capital de
Antioquia cuenta con el Observatorio De Educación Superior De Medellín
``ODES'' el cual se encarga entre muchas cosas del monitoreo y
seguimiento de indicadores de educación superior como lo es la deserción
estudiantil \citep{AlcaldiadeMedellin2019}. El ODES retomando
información del SPADIES hace un análisis de la deserción estudiantil a
nivel Medellín en el cual se registra una tasa de deserción por periodo
del 11,9 \% para el año 2015 \citep{SPADIES2017}. Dentro de la región
antioqueña la Universidad Nacional de Colombia Sede Medellín ocupa el
puesto 19 en tasas de deserción, registrando históricamente un índice de
deserción por cohorte de 50,1 \% al año
2010\citep{UniversidadNacionaldeColombiaSedeMedellin.OficinadePlaneacion2011}.

\hypertarget{metodologia}{%
\section{Metodología}\label{metodologia}}

\hypertarget{recoleccion-de-informacion}{%
\subsection{Recolección de
Información}\label{recoleccion-de-informacion}}

Los datos para realizar el análisis descriptivo de la percepción de
deserción estudiantil fueron recolectados alrededor de todo el campus
universitario utilizando un cuestionario de 35 preguntas, las cuales
incluían variables continuas y categóricas. La pregunta objetivo en
referencia a la interrupción de los estudios universitarios es una
variable categórica lo que dirige el análisis a la estimación de una
proporción. Para la estimación del parámetro de interés se usó un
muestreo aleatorio simple con un nivel de confianza del 95\%, un límite
en el error de estimación de 0,05 y una varianza de 0,25 correspondiente
a una proporción de 0,5. El tamaño de la población empleado fue obtenido
de la información disponible en la oficina de planeación de la sede
Medellín en el cual se reportan 10.594 estudiantes de pregrado al
semestre 2018-1
\citep{UniversidadNacionaldeColombiaSedeMedellin.OficinadePlaneacion2018}.
Para el cálculo del tamaño de muestra necesario para realizar
estimaciones correctas se utilizó la aplicación Shiny del semillero de R
de la escuela de estadística de la Universidad Nacional de Colombia sede
Medellín \citep{SemillerodeR.EscueladeEstadistica2019}, cuyo resultado
fue un tamaño de muestra mínimo de 371 estudiantes a encuestar.

\hypertarget{analisis-de-datos}{%
\subsection{Análisis de datos}\label{analisis-de-datos}}

Para el análisis de datos se utilizaron métodos descriptivos y
exploratorios de las preguntas, se realizaron distintas agrupaciones y
gráficos de apoyo utilizando el programa de libre acceso R \citep{R}.
Para el manejo, selección y ajustes de los datos se utilizaron los
paquetes ``tidyr'' \citep{Wickham2019a} y ``dplyr'' \citep{Wickham2019}.
Los gráficos de dispersión y las curvas ajustadas de regresión además de
ser realizadas con el paquete base se elaboraron también con el paquete
``ggplot2'' \citep{Wickham2016}. Para visualizar y evidenciar la
correlación de algunas de las variables se utilizó el paquete
``corrplot'' \citep{Wei2017}. En algunos casos se mezclaron gráficos en
una misma pantalla utilizando el paquete ``gridExtra''
\citep{Auguie2017}. Los árboles de decisión que se constituyen como una
técnica estadística para la segmentación, la estratificación, la
predicción, la reducción de datos y el filtrado de variables, la
identificación de interacciones, la fusión de categorías y la
discretización de variables continuas \citep{BerlangaSilvente2013},
fueron realizados con el paquete ``rpart'' \citep{Therneau2019}{]}, para
graficarlos se utilizó el paquete ``rpart.plot'' \citep{Milborrow2019}.
Se generaron bosques aleatorios como una alternativa de regresión no
paramétrica la cual se fundamenta en la inferencia condicional
utilizando el paquete ``party'' \citep{Hothorn2019}.

\hypertarget{resultados}{%
\section{Resultados}\label{resultados}}

\begin{figure}

{\centering \includegraphics{Plantilla_trabajos_Rday_files/figure-latex/mifig1-1} 

}

\caption{Frecuencia preguntas dicotómicas “si” y  “no”, por genero del total de encuestados. }\label{fig:mifig1}
\end{figure}

La figura \ref{fig:mifig1} describe las preguntas dicotómicas basadas en
las respuestas ``si'' y ``no'', se realizó un conteo por respuesta y
dentro de cada respuesta por genero de los encuestados. \newline

\begin{figure}

{\centering \includegraphics{Plantilla_trabajos_Rday_files/figure-latex/mifig2-1} 

}

\caption{Intensidad horaria de creditos por semana, linea verde horas semanales del promedio de credtios inscritos, linea amarilla horas promedio de creditos mas horas de ocio web, linea roja horas promedio de creditos mas horas de ocio web y horas de sueño }\label{fig:mifig2}
\end{figure}

La figura \ref{fig:mifig2} describe las horas de dedicación académica
por créditos inscritos, la línea verde representa las horas semanales
dedicadas por el promedio de créditos calculado con los datos obtenidos,
la línea amarilla representa el valor anteriormente calculado añadiendo
las horas de ocio web en la semana y por ultimo la línea roja representa
las horas en la semana dedicadas a estudio, ocio web y dormir para el
promedio de créditos inscritos obtenido en el presente estudio.

\hypertarget{elaboracion-de-tablas}{%
\section{Elaboración de tablas}\label{elaboracion-de-tablas}}

Para construir una tabla se deben tener en cuenta las siguientes
recomendaciones.

\begin{itemize}
\tightlist
\item
  Los números van centrados siempre que tengan la misma cantidad de
  dígitos, de lo contrario deben ir alineados con el margen derecho del
  título.
\item
  Cuando las tablas tienen datos con cifras decimales, el número de
  éstas debe ser igual dentro de la misma columna, pudiendo variar de
  columna a columna.
\item
  Las tablas se deben nombrar en la parte superior.
\end{itemize}

La Tabla \ref{tabla:sencilla} mostrada a continuación se elaboró usando
instrucciones usuales de LaTex.

\begin{table}[htbp]
\begin{center}
\caption{Nombre de la tabla completo.}
\begin{tabular}{l|l} \hline
Pais & Ciudad \\ \hline
Espana & Madrid \\ 
Espana & Sevilla \\ 
Francia & Paris \\ \hline
\end{tabular}
\label{tabla:sencilla}
\end{center}
\end{table}

\hypertarget{incluyendo-codigo-de-r}{%
\section{Incluyendo código de R}\label{incluyendo-codigo-de-r}}

Es posible incluir código de R en este documento para ilustrar a los
lectores en la forma de usar R para realizar algún procedimiento. Se
recomienda a los autores
\href{https://bookdown.org/yihui/rmarkdown/r-code.html}{visitar este
enlace} para conocer más detalles de como incluir código de R. \newline

A continuación un muestra un código de R que genera cien observaciones
aleatorias de una normal y luego calcula la media muestral \(\bar{x}\).

\begin{Shaded}
\begin{Highlighting}[]
\NormalTok{x <-}\StringTok{ }\KeywordTok{rnorm}\NormalTok{(}\DataTypeTok{n=}\DecValTok{100}\NormalTok{, }\DataTypeTok{mean=}\DecValTok{70}\NormalTok{, }\DataTypeTok{sd=}\DecValTok{5}\NormalTok{)}
\KeywordTok{mean}\NormalTok{(x)}
\end{Highlighting}
\end{Shaded}

\begin{verbatim}
## [1] 70.03231
\end{verbatim}

\hypertarget{elaboracion-de-figuras}{%
\section{Elaboración de figuras}\label{elaboracion-de-figuras}}

Toda figura debe ir centrada y el título debe ir en la parte inferior

\begin{figure}
\centering
\includegraphics{Plantilla_trabajos_Rday_files/figure-latex/mifig231-1.pdf}
\caption{Histograma de valores simulados de una N(0, 1)}
\end{figure}

\hypertarget{ecuaciones}{%
\section{Ecuaciones}\label{ecuaciones}}

Para incluir ecuaciones dentro de un párrafo se usan \texttt{\$\ \$} y
dentro el símbolo deseado. Por ejemplo, para incluir la letra griega
\(\mu\) se escribió \texttt{\$\textbackslash{}mu\$}. \newline 

Para incluir ecuaciones se pueden usar \texttt{\$\$\ \$\$} y dentro la
ecuación. La siguiente ecuación fue hecha usando
\texttt{\$\$\textbackslash{}mu\ =\ \textbackslash{}frac\{\textbackslash{}theta\}\{\textbackslash{}sigma\}\$\$}.

\[\mu = \frac{\theta}{\sigma}\]

Otra forma de incluir ecuaciones es de la forma usual como se hace en
latex, a continuación un ejemplo usando
\texttt{\textbackslash{}begin\{equation\}} y
\texttt{\textbackslash{}end\{equation\}}. La ventaja de esta última
opción es que la ecuación \ref{energia} sale numerada y se puede citar
luego usando \texttt{\textbackslash{}ref\{clave\}}.

\begin{equation} \label{energia}
E = m c^2
\end{equation}

\hypertarget{incluyendo-referencias}{%
\section{Incluyendo referencias}\label{incluyendo-referencias}}

Todas las referencias deben estar en el archivo \texttt{master.bib} para
que se puedan invocar en el trabajo. Existen dos formas de citar y son:

\begin{enumerate}
\def\labelenumi{\arabic{enumi}.}
\tightlist
\item
  Como cita directa, es decir como \citet{xie2013ddrk}. Para esto se
  debe escribir \texttt{@xie2013ddrk} y el archivo insertará
  automaticamente la referencia y la colocará en la última sección de
  Referencias.
\end{enumerate}




\newpage
\singlespacing 
\end{document}
